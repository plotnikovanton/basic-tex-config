% Russian language support
\usepackage{mathtext}
\frenchspacing


% Tables
\usepackage{tabularx}

\usepackage{indentfirst}
\usepackage{multicol}

% Work with images
\usepackage{graphicx}
\usepackage{wrapfig}

% Improved mathematic features
\usepackage{amsmath,amsfonts,amssymb,amsthm,mathtools}
\usepackage{icomma}
\usepackage{dsfont} % Badass mathematical font
%\renewcommand{\arraystretch}{1.5} % Make formulas more readable
% \colvec[scale]{data} produces vector column which looks smaller than matrix
\newcommand{\colvec}[2][.8]{%
  \scalebox{#1}{$\begin{array}{@{}c@{}}#2\end{array}$}}%

% Colors
\usepackage{color}

% Some typographic features
\usepackage{setspace}

% References
\usepackage[hidelinks]{hyperref}
\usepackage{cleveref}
\usepackage{fancyref}
\usepackage{autonum}


% Theorems
\theoremstyle{remark}
\newtheorem{remark}{Замечание}

\usepackage{ifxetex}
\ifxetex
    %%% Math font
    \usepackage{mathspec}
    \defaultfontfeatures{Ligatures=TeX}
    %\setmathsfont(Digits){TeX Gyre Termes}
    %\setmathsfont(Latin){TeX Gyre Termes}
    %\setmathrm{TeX Gyre Termes}

    \usepackage[no-math]{fontspec}
    \setmainfont{Linux Libertine O}

    \usepackage{polyglossia}
    \setmainlanguage{russian}
    \setotherlanguage{english}
\else
    \usepackage{cmap}
    \usepackage[T2A]{fontenc}
    \usepackage[utf8]{inputenc}
    \usepackage[english,russian]{babel}
\fi

