% Author: Plotnikov Anton
% 2016
%
% My simple latex preamble feel free to modify and use
%

% Russian polygraph's style support
\usepackage{indentfirst} % indentation on first paragraph in section
\frenchspacing % tells LaTeX not to insert extra space at the end of sentences
\usepackage{setspace}

% Tables
\usepackage{tabularx}
\usepackage{multicol}

% Work with images
\usepackage{graphicx} % support of images
\usepackage{wrapfig} % support of wrapping figures by text
\usepackage{float} % let you insert images where you really want

% Improved mathematic features
\usepackage{mathtext}
\usepackage{braket}
\usepackage{amsmath,amsfonts,amssymb,amsthm,mathtools}
\usepackage{mathrsfs}
\usepackage{icomma}
\usepackage{dsfont} % Badass mathematical font

% Colors
\usepackage{color}

% References
\usepackage[hidelinks]{hyperref}
\usepackage{cleveref}
\usepackage{fancyref}
\usepackage{autonum}


% Theorems
\newtheorem{theorem}{Теорема}
\theoremstyle{remark}
\newtheorem{remark}{Замечание}

% Defines Russian localization for xetex or other environment separately
\usepackage{ifxetex}
\ifxetex%
    \usepackage{mathspec}
    \usepackage{fontspec}
    %\setmainfont{Times New Roman}
    \setmainfont{Liberation Serif}
    \setmathfont{TeX Gyre Pagella Math}

    \usepackage{polyglossia}
    \usepackage{csquotes}
    \newfontfamily{\cyrillicfonttt}{Liberation Serif}
    %\newfontfamily\cyrillicfonttt{Linux Libertine 0}
    \setmainlanguage{russian}
    \setotherlanguage{english}
    \setotherlanguage[variant=us]{english}
    \defaultfontfeatures{Ligatures={TeX},Renderer=Basic} 
\else
    \usepackage{cmap}
    \usepackage[T2A]{fontenc}
    \usepackage[utf8]{inputenc}
    \usepackage[english,russian]{babel}
\fi

